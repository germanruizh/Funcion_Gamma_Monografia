\chapter*{Introducción}
\addcontentsline{toc}{chapter}{Introducción}
\chaptermark{Introducci\'on}

Una de las exposiciones más completas y claras que se han hecho sobre la función gamma fue escrita por Artin (1961), con el motivo de suplir el poco tratamiento realizado en los textos elementales de cálculo y análisis. De igual manera, en este trabajo se trata de remediar este hecho, ya que en la literatura en español no se da un tratamiento claro y profundo de esta función, y si existen trabajos académicos al respecto, no son cercanos a la literatura que maneja un estudiante de primer o segundo año de ciencias matemáticas. 

Siguiendo el trabajo hecho por Lima (1996) sobre los logaritmos, se tuvo una base para la construcción del presente trabajo. El profesor Lima realiza una presentación clara y concisa de una de las funciones más importantes y comunes de la matemática elemental, basándose en su origen y desarrollo histórico, en las diferentes representaciones prácticas y sus propiedades, y lo más importante, en su utilidad. Empezando con un acercamiento histórico a la definición de logaritmo, llega de manera intuitiva a una representación formal del mismo, dándole un carácter de rigurosidad en tal exposición. Al contrario de Arens (2013), el cual trabaja a partir de la definición de función exponencial compleja representada por medio de una serie de potencias, el profesor Lima busca un tratamiento más cercano al nivel de las matemáticas de bachillerato o de primer año de universidad.

Al contrario de muchas representaciones usuales de la función gamma, incluyendo la realizada por Artin, se busca una exposición a partir del origen histórico de dicha función. Por tal razón, se prefirió comenzar con el producto infinito dado por Euler (1738), que en Davis (1959) se expone de manera breve. 

Siguiendo con el deseo de dar una presentación agradable de la función gamma de Euler, en el primer capítulo se introduce el problema abierto enunciado por Daniel Bernoulli y Christian Goldbach sobre la interpolación de los términos que surgían de la fórmula factorial. En solo 2 cartas, Euler solucionó dicho problema que dio origen a las funciones gamma y beta, las cuales tiempo después adquirieron su definición moderna. En el capítulo segundo se presenta la representación de producto infinito (dado por Euler y luego definido por Gauss en el plano complejo) en la recta real $\mathbb{R}$, se determina su dominio de definición y se demuestra su continuidad siguiendo el tratamiento hecho por Fischer (1983). Luego de anotar algunas características notables de la función, se demuestra el teorema de Bohr-Mollerup y se deduce su representación integral siguiendo el tratamiento hecho por Fischer (1983) y Artin (1961). Al final del capítulo se deducen la fórmula de Stirling para la aproximación del factorial y la fórmula de multiplicación de Gauss, terminando con varias propiedades relacionadas con la función beta y la función seno.

En el capítulo tercero, partiendo de la forma integral de la función gamma, se extiende su dominio al plano complejo $\mathbb{C}$. Primero, trabajando con el semiplano derecho $\{z \in \mathbb{C} : Re(z)>0 \}$, donde se muestra que es holomorfa en este dominio. Luego, se realiza su prolongación analítica al conjunto $\mathbb{C}\texttt{\textbackslash}(\mathbb{Z}^- \cup \{0\})$ junto con la representación de Weierstrass y la deducción de algunas de sus propiedades importantes análogas a las vistas en el capítulo 2. Por último, se demuestra el teorema de Wielandt siguiendo a Remmert (2007), el cual es una generalización del teorema de Bohr-Mollerup pero en el plano complejo. El capítulo se termina con las representaciones de las funciones gamma y beta como integrales de contorno.

El cuarto y último capítulo trata de las aplicaciones escogidas para este texto. Primero, se generaliza el teorema del binomio de Newton junto con la fórmula del volumen de una $n$-esfera. Luego, se deduce de una manera muy natural la derivada fraccionaria, terminando así con la definición de la derivada de Riemann-Liouville y algunas propiedades de ella. Las 2 aplicaciones siguientes tienen que ver con la teoría analítica de números y la ciencia estadística. En particular, se ha trabajado con la función zeta de Riemann y con la distribución gamma, destacando sus características más notables. Por último, para finalizar el presente trabajo de grado, se realiza la aplicación de la función gamma en la definición de las funciones de Bessel y de la transformada de Mellin.

Se ha hecho un gran esfuerzo en recopilar toda la información para la elaboración de este documento. Esperamos que sea de su agrado y que motive al lector a aventurarse en el estudio de las funciones especiales, ya que este tema es de gran importancia en la matemática pura y aplicada.
\endinput